\documentclass[11pt]{kcw-protocol}

\usepackage[ngerman]{babel}
\usepackage{blindtext}

\title{Protokoll der Mitgliederversammlung vom 24. Dezember 1809}

\versammlungsort{auf der Zugspitze}
\versammlungsbeginn{19:00}
\versammlungsende{19:01}
\anwesende{sh. Liste im Anhang, etwa 7 stimmberechtigte Mitglieder}
\versammlungsleitung{Valentin Versammlungsleiter}
\protokoll{Peter Protokollant}

\begin{document}

\maketitle

\tableofcontents

\section{Begrüßung}

\blindtext

\anwesenheit{Jemand hatte keine Lust mehr, den Blindtext zu lesen. Dieses Makro
  ist dazu da, solche An- und Abwesenheitsbemerkungen festzuhalten.}

\blindtext

\section{Zwei Abstimmungen}

\abstimmung{Die nächste Abstimmung soll keine automatisch erzeugte Bemerkung
  haben, ob ihr Vorschlag mit einfacher Stimmenmehrheit angenommen
  ist.}{14}{15}{6}

\abstimmungMitBemerkung{Der Vorschlag der letzten Abstimmung hätte nicht
  umgesetzt werden dürfen.}{3}{1}{66}{Die vielen Enthaltungen zeigen, dass
  wenige Logiker in der Versammlung sind.}

\section{Ein Streitpunkt}

Nach langer Diskussion entscheiden wir uns für ein Meinungsbild. Es gibt ganz
viele verschiedene Meinungen.

\meinungsbild{Was soll's zu Essen geben?}
{Lasagne & 2 \\
  Lasagne mit viel Käse & 14 \\
  keine Lasagne, aber Käse & 5\\
  Käsebrot, aber auch Lasagne & 1}
{Das Meinungsbild bekommt eine Abschlussbemerkung, in der wir zusammenfassen,
  was dem Vorstand hiermit aufgetragen werden soll.}

\unterschriften{Walter Wichtig, Vorstandsvositzender}
  {Wendelin Wenigerwichtig, Schriftführer}
  {Eine zufällige Person, die nicht einmal Mitglied des Vereins ist}

\appendix

\section{Anwesenheitsliste}

\section{Folien der Präsentation}

Die kann man zum Beispiel mit \verb|pdfpages| einbinden.


\end{document}